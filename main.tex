\documentclass{article}
\title{Compsci 590T Homework 8}
\usepackage[margin=1in]{geometry}
\usepackage[loose,nice]{units}
\usepackage{enumitem}
\usepackage{amsmath}
\usepackage{multirow,array}
\newcommand{\pvec}[1]{\vec{#1}\mkern2mu\vphantom{#1}}
\usepackage{amssymb}
\usepackage{cancel}
\usepackage{mathabx}
\usepackage{hyperref}
\newcommand\tab[1][1cm]{\hspace*{#1}}
\usepackage{tikz}
\newcommand*\circled[1]{\tikz[baseline=(char.base)]{
            \node[shape=circle,draw,inner sep=2pt] (char) {#1};}}

\usetikzlibrary{positioning}
\tikzset{main node/.style={circle,fill=blue!20,draw,minimum size=4mm,inner sep=0pt},}
\author{Student: Brian Sun, Collaborator: Thai On}
\date{}

\begin{document}
\maketitle


\section{Problem 1}
Assume player $i$ (players who values all rooms equally) is not always pay the least amount of rent in any envy-free outcome. If this is true, there will exist an envy-free outcome that at least one player $i'$ who don't value all room the same and pay less than $i$.\\\\
Therefore the room $i'$ get is cost less than the room $i$ get.\\
However this violate the evny-free property: $v_{i\sigma(i)} - p_{\sigma(i)} \geq v_{ij} - p_j,\ \forall j$.\\
Since $i$ values all rooms equally and the room $i'$ get is cost less than the room $i$ get. \\
So $p_{\sigma(i)} > p_k \tab$(define room $k$ is the room player $i'$ get)\\
which leads to $v_{i\sigma(i)} - p_{\sigma(i)} \not\geq v_{ik} - p_k \tab$\\\\
So the statement "player $i$ (players who values all rooms equally) is not always pay the least amount of rent in any envy-free outcome" is false. Therefore player $i$ pays the least amount of rent in any envy-free outcome, since if there exist an player who pay less than $i$, than it's not envy-free anymore.

%%%%%%%%%%%%%%%%%%%%%%%%%%%%%%%%%%%%%
\newpage
\section{Problem 2}

Base on the envy-free property we have: $\overbrace{v_{i\sigma(i)}- p_{\sigma(i)}}^{A} \geq \overbrace{v_{ij} - p_j}^{B},\ \forall j$\\
define $A = v_{i\sigma(i)}- p_{\sigma(i)}$ and $B = v_{ij} - p_j$\\
To satisfy envy-free there will have two different outcome:\\
1: when $A>B$\\
2: when $A=B$\\\\
For case 1:\\
if $A > B$ that means there exist an room $j^*$ such that $v_{i\sigma(i)}- p_{\sigma(i)} > v_{ij^*} - p_{j^*}$\\
since room $j^*$ will assign to a player $i'$ and because everyone have the same valuation for all the rooms so $i' = i$.\\
Therefore we have $v_{i'j^*} - p_{j^*} < v_{i\sigma(i)} - p_{\sigma(i)}$,
since $v_{i'j^*} - p_{j^*} \not \geq v_{i\sigma(i)} - p_{\sigma(i)}$ which break the envy-free property.
Therefore case 1 cannot happen, since when $A > B$, then the outcome is not envy-free.\\\\
For case 2:\\
since base on case 1 we know $A > B$ cannot satisfy envy-free, so we have to assume $A = B$ for all players, so the outcome satisfy envy-free.\\\\
when $A = B$ there will have three cases:\\
$a$: $A = B > 0$\\
$b$: $A = B < 0$\\
$c$: $A = B = 0$\\\\
For case $(a)$:\\
if $A = B > 0$ happen, that means $v_{i\sigma(i)} > p_{\sigma(i)},\ \forall i$\\
Therefore $\sum_{i = 1}^n v_{i\sigma(i)} > \sum_{i = 1}^n p_{\sigma(i)}$\\\\
$\sum_{i}^n v_{i\sigma(i)}$ means the sum of all the player value their assign room. Since there are $n$ players and $n$ rooms and everyone cannot get the same room, therefore $\sum_{i = 1}^n v_{i\sigma(i)} = \sum_{j = 1}^n v_{ij}$\\
$\sum_{i = 1}^n p_{\sigma(i)}$ means the sum of all the room price that assign to players. Since there are $n$ players and $n$ rooms and everyone cannot get the same room, therefore $\sum_{i = 1}^n p_{\sigma(i)} = \sum_{j = 1}^n p_{j}$\\\\
Since $\sum_{i = 1}^n v_{i\sigma(i)} = \sum_{j = 1}^n v_{ij}$ and $\sum_{i = 1}^n p_{\sigma(i)} = \sum_{j = 1}^n p_{j}$ and $\sum_{i = 1}^n v_{i\sigma(i)} > \sum_{i = 1}^n p_{\sigma(i)}$\\
so we got $\sum_{j = 1}^n v_{ij} > \sum_{j = 1}^n p_{j}$\\
Which this break the requirement from the question: $\sum_{j = 1}^n v_{ij} = \sum_{j = 1}^n p_{j}$\\
Since case $a$ break the requirement from the question, therefore case $a$ cannot happen too.\\\\
For case $(b)$:\\
if $A = B > 0$ happen, that means $v_{i\sigma(i)} < p_{\sigma(i)},\ \forall i$\\
Therefore $\sum_{i = 1}^n v_{i\sigma(i)} < \sum_{i = 1}^n p_{\sigma(i)}$\\\\
Since $\sum_{i = 1}^n v_{i\sigma(i)} = \sum_{j = 1}^n v_{ij}$ and $\sum_{i = 1}^n p_{\sigma(i)} = \sum_{j = 1}^n p_{j}$ and $\sum_{i = 1}^n v_{i\sigma(i)} < \sum_{i = 1}^n p_{\sigma(i)}$\\
so we got $\sum_{j = 1}^n v_{ij} < \sum_{j = 1}^n p_{j}$\\
Which this break the requirement from the question: $\sum_{j = 1}^n v_{ij} = \sum_{j = 1}^n p_{j}$\\
Since case $b$ break the requirement from the question, therefore case $b$ cannot happen too.\\\\
For case $(c)$:\\
if $A=B=0$ happen, that means $v_{i\sigma(i)} = p_{\sigma(i)},\ \forall i$\\
Therefore $\sum_{i = 1}^n v_{i\sigma(i)} = \sum_{i = 1}^n p_{\sigma(i)}$\\\\
Since $\sum_{i = 1}^n v_{i\sigma(i)} = \sum_{j = 1}^n v_{ij}$ and $\sum_{i = 1}^n p_{\sigma(i)} = \sum_{j = 1}^n p_{j}$ and $\sum_{i = 1}^n v_{i\sigma(i)} = \sum_{i = 1}^n p_{\sigma(i)}$\\
so we got $\sum_{j = 1}^n v_{ij} = \sum_{j = 1}^n p_{j}$.\\
Since case $(c)$ is the only case that satisfy envy-free and $\sum_{j = 1}^n v_{ij} = \sum_{j = 1}^n p_{j}$. Therefore there exist only one unique EF price vector. That is $p_{\sigma(i)} = v_{i\sigma(i)},\ \forall i$ which is set all the room price equal to player's room valuation. 
%%%%%%%%%%%%%%%%%%%%%%%%%%%%%%%%%%%%%%%
\newpage
\section{Problem 3}

If this is not true that means there exist an room $j^*$ such that $p_{j} > p_{j^*}$ and the outcome is envy-free.\\
Base on the given information we know every player value room $j$ the least, therefore $v_{ij} < v_{ij^*}$\\\\
Assume player $i$ got assign to room $j$
\\therefore base on the envy-free property we know $v_{ij} - p_j \geq v_{ij^*} - p_{j^*} \longrightarrow (\star)$\\\\
However since $p_{j} > p_{j^*} = -p_{j} < -p_{j^*}$ and  $v_{ij} < v_{ij^*}$ we know $v_{ij} - p_j < v_{ij^*} - p_{j^*}$ which contradict with $(\star)$.\\\\
Therefore the statement "there exist an room $j^*$ such that $p_{j} > p_{j^*}$ and the outcome is envy-free" is false. So the price of room $j$ should be the lowest, since if there are room lower that $j$ than it's not envy-free anymore.


%%%%%%%%%%%%%%%%%%%%%%%%%%%%%%%%%%%%%%%%
\newpage
\section{Problem 1.a}
Base on the VCG mechanism determine the payment for each player we have:
$$p_i = \sum_{j \not = i} v_j(o_{-i}^*) - \sum_{j \not = i}v_j(o^*) = \sum_{j \not = i}[v_j(o_{-i}^*) - v_j(o^*)] \longrightarrow (\star)$$
Define $S^*$ as the set contain all the successful bidders from $o^*$. Another way to say this is $S^*$ has all the bidders form $o^*$ who's utility is greater than $0$.\\\\
Since there are $k$ items and each bidder only can have one item, therefore both $S^*$ will have the size of $k$ bidders.\\\\
Base on VCG mechanism, $o^*_{-i}$ is the socially optimal outcome without $i$. Therefore given any $i$ was in $S^*$ that means we have to make a new set $S^*_{-i}$ to find the substitute bidder $i'$ for $i$.\\
Since we are making the socially optimal outcome, therefore $i'$ should be the highest bidder amount all the bidders who are not in $S^*$. Because if $i'$ is not the highest bidder amount all the bidders who are not in $S^*$, then it's not socially optimal outcome anymore.\\\\
Base on $(\star)$ we know how we assign the value for these successful bidders is base on the sum of $v(j) \in S^*_{-i}$ minus the sum of $v(j) \in S^* \setminus i$ $\tab \bullet$($j$ is a player)\\
Since the only different between $S^*$ and $S^*_{-i}$ is $S^*$ have $i$ but don't have $i'$ and $S^*_{-i}$ have $i'$ but don't have $i$. Therefore the only different between $S^*\setminus i$ and $S^*_{-i}$ is $S^*_{-i}$ have $i'$ but $S^*$ don't have.\\\\
Since the only different between $S^*\setminus i$ and $S^*_{-i}$ is $S^*_{-i}$ have $i'$ but $S^*$ don't have. Therefore the sum of $v(j) \in S^*_{-i}$ minus the sum of $v(j) \in S^* \setminus i$ is always equal to $v(i')$\\\\
Since no matter which bidder we take away from $S^*$ VCG will always find the same bidder $i'$ to substitute $i$. Therefore all the successful bidder will have the same pay, since base on $(\star)$ the formula will always return $v(i')$.

%%%%%%%%%%%%%%%%%%%%%%%%%%%%%%%%%%%%%%%%
\newpage
\section{Problem 1.b}
 It is possible for example: \\
 Assume the auction have 4 bidders and 3 items A,B and C.\\
 set A and B are identical\\
 then we can make a table of each player bids the items.
 \begin{center}
\begin{tabular}{ | m{4em} | m{4em}| m{4em} | m{4em} |} 
  \hline
   & A,B & C \\ 
  \hline
  $p_1$ & 100 & 100 \\ 
  \hline
  $p_2$ & 100 & 200 \\ 
  \hline
  $p_3$ & 200 & 100 \\ 
  \hline
  $p_4$ & 50 & 100 \\ 
  \hline
\end{tabular}
\end{center}
Base on the VCG mechanism we know $o^* = 500$ with $p_1$ get item A, $p_2$ get item C and $p_3$ get item B.\\
set $W^+_i = \sum_{j \not = i} v_j(o^*)$\\
set $W^-_i = \sum_{j \not = i} v_j(o_{-i}^*)$\\\\
Base on VCG mechanism we know they way it determine the payment for bidder $i$ will be:
\begin{align*}
&W^+_1 = 400\\
&W^-_1 = 450\\
&p_1 = 450 - 400 = 50\\\\
&W^+_2 = 300\\
&W^-_2 = 400\\
&p_2 = 400 - 300 = 100\\\\
&W^+_3 = 300\\
&W^-_3 = 350\\
&p_3 = 350 - 300 = 50\\
\end{align*}
Since bidder $2$ and $3$ receive items they value the same ($p_2$ get item C and $p_3$ get item B). But $p_3$ pay $50$ and $p_2$ pay $100$. Therefore it is possible that two bidders who receive items they value the same will pay different prices.

%%%%%%%%%%%%%%%%%%%%%%%%%%%%%%%%%%%%%%%%
\newpage
\section{Problem 2}

VCG cannot extract truthful valuations from the bidders.\\ 
For example there exist cases such as:\\
$v_1 > v_2 > v_3 > ... v_n$\\
In this case if $p_1$ works with $p_2$, they can maximize their joint utility by let $p_2$ say $v_2 = 0$ therefore base on the valuation, we will have  $v_1 > v_3 > ... v_n$\\
Base on this valuation, VCG will let $p_1$ be the winner of the auction and let $p_1$ pay the second highest price, which is $v_3$. Since this is the single item auction, therefore everyone except $p_1$ utility will be $0$ and $p_1$ utility will be $u_1 = v_1 - v_3$. Therefore $p_1$ and $p_2$ joint utility will be $u_1 + u_2 = v_1 - v_3 + 0$\\\\
Which compare to if they don't work together, $p_1$ will be the winner of the auction. $p_1$ will still be the winner, since $p_1$ is the highest and base on VCG, $p_1$ will pay the second price which is $v_2$. Since this is the single item auction therefore everyone's utility will be $0$ except $p_1$'s utility will be: $u_1 = v_1 - v_2$. $p_1$ and $p_2$ joint utility will be $v_1 - v_2 + 0$\\\\
Since $v_3 < v_2$ therefore $v_1 - v_2 + 0 < v_1 - v_3 + 0$ Which means $p_1$ and $p_2$ can have a higher joint utility when they work together. therefore VCG cannot extract truthful valuations from the bidders in this case.
%%%%%%%%%%%%%%%%%%%%%%%%%%%%%%%%%%%%%%%%
\newpage
\section{Problem 3.a}

Third price auction is not truthful.
If the third price auction is truthful, that means all bidders best response is always to bid truthfully. however this is not truth base on the case where the second highest bidder $i$ bidding honestly or dishonestly will win or lose the auction.\\\\
If $i$ bidding honestly, and $i$ is the second highest bidder, then $i$ cannot win the auction since he is not the highest bidder, therefore $i$'s utility is $0$.\\
If $i$ bidding untruthfully by increase his value higher the the highest bidder, than $i$ become the highest bidder and win the auction. Since $i$ win the auction so he will pay the third highest prise. Since $i$'s true value for the item is higher than the third highest bidder, therefore it is good for $i$ since he win the auction and pay less than what he actually think the item is.\\\\
Therefore third price auction is not truthful, since the second highest bidder can lie his bid to be highest and win the auction without paying more than what he actually values it.

%%%%%%%%%%%%%%%%%%%%%%%%%%%%%%%%%%%%%%%%
\newpage
\section{Problem 3.b}

This auction is truthful since players lie to increase or decrease their value will not increase their utility.\\\\
To prove this is true, we are going to look at:\\
1. when player $i$ win by telling the truth\\
2. when player $i$ lose by telling the truth\\\\
For the first case:\\
when player $i$ win, that means $i$ is going to pay the second highest bid, or the reserve price (whichever is higher).\\
If $i$ lie by increase his bid, this action will not effect the second highest bid or the reserve price, so he is still going to pay $max(b_2,r)$. Which means his utility still the same $u_i = v_i - max(b_2,r)$.\\
If $i$ lie by decrease his bid, if $b_i$ still greater than $max(b_2,r)$, then this action will still not effect the second highest bid or the reserve price. So he is still going to pay $max(b_2,r)$ which means his utility still the same $u_i = v_i - max(b_2,r)$.\\
If $i$ lie by decrease his bid, if $b_i$ less or equal to $max(b_2,r)$, then this action will change the fact that $i$ is the winner, since $b_1 \leq max(b_2,r)$. Since $i$ is not the winner anymore therefore his utility will from  $u_i = v_i - max(b_2,r)$ to $u_i = 0$ which is worse than before.\\
In the first case, since no matter how $i$ lie by increase or decrease the value his utility will either be the same or become worse, therefore there is no reason for bidder $i$ to lie.\\\\
For the second case:\\
when player $i$ lose the auction, that means it's either: \\
a. $b_i$ is the highest bid but less than $r$\\
b. $b_i$ is not the highest bid which $b_i < max(b_1,r)$\\\\
For case a: \\
If $i$ increase his bid higher but still not greater than $r$, than this action will not change the fact that he lose the auction. Since $i$ still lose the auction, so his utility will still be $0$ like before.\\ 
If $i$ increase his bid higher than $r$, than this action will make him become the winner, but this will decrease his utility since his true valuation is $v_i < r$ and now $i$ has to pay $r$ therefore his utility will be $u_i = v_i - r < 0$ which is worse than before.\\
If $i$ decrease his bid, than this action ill not change the fact that he lose the auction. Since $i$ still lose the auction, so his utility will still be $0$ like before.\\\\
For case b:\\
If $i$ increase his bid higher but will not greater than $max(b_1,r)$ than this action will not change the fact that he lose the auction. Since $i$ still lose the auction, so his utility will still be $0$ like before.\\
If $i$ increase his bid higher than $max(b_1,r)$, than this action will make him become the winner, but this will decrease his utility since his true valuation is $v_i < r$ and now $i$ has to pay $r$ therefore his utility will be $u_i = v_i - max(b_1,r) < 0$ which is worse than before.\\
If $i$ decrease his bid, than this action ill not change the fact that he lose the auction. Since $i$ still lose the auction, so his utility will still be $0$ like before.\\\\
In the second case, since no matter how $i$ lie by increase or decrease the value, his utility will either be the same or become worse, therefore there is no reason for bidder $i$ to lie.\\\\
Since no matter the player win or lose they cannot increase their utility by give a wrong valuation, therefore this auction is truthful.

%%%%%%%%%%%%%%%%%%%%%%%%%%%%%%%%%%%%%%%%
\newpage
\section{Problem 4}
Assume bidder $i'$ bid $\frac{n-1}{n}v_{i'}$ and bidder $i$ bids $s_i$\\
$i$ will win when $v_{i'} < \frac{n-1}{n}s_i$ and will have the utility of $v_i - s_i$, but will lose when $v_{i'} >  \frac{n-1}{n}s_i$ and will have the utility $0$.
Base on these information we can have expected probability for player $i$: 
\begin{align*}
E[u_i] &= \int_0^{\frac{n-1}{n}s_i} (v_r - s_r) dv_{i'} + \int_{\frac{n-1}{n}s_i}^1 (0) dv_{i'}\\
&= \frac{n}{n-1}s_iv_i - \frac{n}{n-1}s_i^2
\end{align*}
Find player $i$ best bid by derivative the result:
\begin{align*}
\frac{d}{ds_i}(\frac{n}{n-1}s_iv_i - \frac{n}{n-1}s_i^2) &= 0\\
\frac{n}{n-1}v_i - 2\frac{n}{n-1}s_i &= 0\\
s_i = \frac{1}{2}v_i
\end{align*}
This apply to when $i'$ bit $s_{i'}$\\
Since this is only for two player game, so $n = 2$
%%%%%%%%%%%%%%%%%%%%%%%%%%%%%%%%%%%%%%%%

\end{document}